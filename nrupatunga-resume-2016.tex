% resume.tex
%
% (c) 2002 Matthew Boedicker <mboedick@mboedick.org> (original author) http://mboedick.org
% (c) 2003-2007 David J. Grant <davidgrant-at-gmail.com> http://www.davidgrant.ca
% (c) 2007-2014 Todd C. Miller <Todd.Miller@courtesan.com> http://www.courtesan.com/todd
%
% This work is licensed under the Creative Commons Attribution-ShareAlike 3.0 Unported License. To view a copy of this license, visit http://creativecommons.org/licenses/by-sa/3.0/ or send a letter to Creative Commons, 171 Second Street, Suite 300, San Francisco, California, 94105, USA.

\documentclass[letterpaper,11pt]{article}

%-----------------------------------------------------------
\usepackage[empty]{fullpage}
\usepackage{color}
\usepackage{xcolor}
\usepackage[colorlinks = true,
            linkcolor = blue,
            urlcolor  = blue,
            citecolor = blue,
            anchorcolor = blue]{hyperref}
\usepackage{hyperref}
\usepackage{xspace}
\usepackage{verbatim}
\usepackage{marvosym}
\usepackage{fontspec}
\usepackage{fontawesome}
\definecolor{mygrey}{gray}{0.80}
\raggedbottom
\raggedright
\setlength{\tabcolsep}{0in}

% Adjust margins to 0.5in on all sides
\addtolength{\oddsidemargin}{-0.5in}
\addtolength{\evensidemargin}{-0.5in}
\addtolength{\textwidth}{1.0in}
\addtolength{\topmargin}{-0.5in}
\addtolength{\textheight}{1.0in}

%-----------------------------------------------------------
%Custom commands
\def\CC{{C\nolinebreak[4]\hspace{-.05em}\raisebox{.4ex}{\tiny\bf ++}}}
%\newcommand{\CC}{C\nolinebreak\hspace{-.05em}\raisebox{.4ex}{\tiny\bf +}\nolinebreak\hspace{-.10em}\raisebox{.4ex}{\tiny\bf +}}
\newcommand*{\Cpp}{C\ensuremath{++}\xspace}
\newcommand*{\phoneB}{{\fontspec{symbola.ttf}\symbol{"260E}}}
\newcommand{\resitem}[1]{\item #1 \vspace{-2pt}}
\newcommand{\resheading}[1]{{\large \colorbox{mygrey}{\begin{minipage}{\textwidth}{\textbf{#1 \vphantom{p\^{E}}}}\end{minipage}}}}
\newcommand{\ressubheading}[4]{
	\begin{tabular*}{7.0in}{l@{\extracolsep{\fill}}r}
		\textbf{#1} & #2 \\
		\textit{#3} & \textit{#4} \\
	\end{tabular*}\vspace{-6pt}}
%-----------------------------------------------------------

\begin{document}

\begin{tabular*}{7.5in}{l@{\extracolsep{\fill}}r}
	\textbf{\Large Nrupatunga}  & \faMobile \hspace{1mm}\emph{+91-9972716433}\\
	&\faGlobe \hspace{1mm} \href{https://nrupatunga.github.io/}{\emph{https://nrupatunga.github.io/}}\\
	&\faEnvelope \hspace{1mm}\emph{nrupatunga.tunga@gmail.com} \\
	%&  \faLinkedin \hspace{1mm}\emph{https://in.linkedin.com/in/nrupatunga} \\
	%&\faGithub \hspace{1mm} \href{https://github.com/nrupatunga}{\emph{https://github.com/nrupatunga}} \\
\end{tabular*}
\\

\vspace{0.1in}
\resheading{Research Interest}
\begin{itemize}
		\resitem{My current interest and focus is on applying machine learning and deep learning techniques 
			to analyze images for the problem of recognition, segmentation and scene understanding. 
			Also, use deep networks to learn and combine features over multiple modalities}
\end{itemize}

\resheading{Education}
\begin{itemize}
	\item
		\ressubheading{Indian Institute of Science}{Bangalore, Karnataka}{Master of Engineering in Signal Processing}{2010 - 2012}
		\begin{itemize}
				\resitem{Master Thesis: Complex Network Approach for Analysis of Biomedical signals}
				\resitem{CGPA: 5.8/8.0}
				\resitem{Advisor: Prof. D. Narayana Dutt}
		\end{itemize}
	\item
		\ressubheading{Sri Jayachamarajendra College of Engineering}{Mysore, Karnataka}{Bachelor of Engineering in Electronics and Communication }{2005 - 2009}
		\begin{itemize}
				\resitem{Percentage: 71.14\%}
		\end{itemize}

\end{itemize}

\resheading{Work Experience}
\begin{itemize}
	\item
		\ressubheading{Samsung R\&D India}{Bangalore}{Technical Lead, Media Analytics and Recognition Team}{2016-Present}
		\vspace{1mm}
		\begin{itemize}
				% Project-1 
				\resitem[]{\faCircleO
					\hspace{1mm}Project: Semantic Segmentation of Sky and Non-sky regions in an Image using Fully Convolutional Neural Network
					\faGlobe \hspace{1mm} \href{https://nrupatunga.github.io/}{\emph{Blog}}}
				\begin{itemize}
						\resitem{Development: Languages \& Tools used - Python, Caffe Deep Learning Framework}
				\end{itemize}
				\begin{itemize}
						\resitem{Aim of this project is to:}
						\begin{itemize}
								\resitem{Understand how Fully Convolutional network enables end to end dense learning}
								\resitem{Fine tune the weights of the pretrained model, appreciate how transfer learning enables 
									to address different computer vision problems with reasonable amount of data}
								\resitem{Investigate the features learnt in each layer of the network}
								\resitem{Experimentation on using sky segmentation map as prior for horizon detection}
						\end{itemize}
				\end{itemize}
				\vspace{1mm}
				% Project-2
				\resitem[]{\faCircleO \hspace{1mm}Project: Nearest Neighbor Image retrieval using GIST descriptor
					\faWrench \hspace{1mm} \href{https://github.com/nrupatunga/GIST-global-Image-Descripor}{\emph{Command Line Tool}}}
				\begin{itemize}
						\resitem{Development: Languages \& Tools used - \CC, OpenCV, MATLAB}
				\end{itemize}
				\begin{itemize}
						\resitem{Aim of this project is to:}
						\begin{itemize}
								\resitem{Evaluate GIST descriptor for task of Image retrieval}
								\resitem{Demonstrate how GIST descriptor can be used for detection of duplicate images}
						\end{itemize}
				\end{itemize}
		\end{itemize}
		\vspace{1mm}
	\item
		\ressubheading{Samsung R\&D India}{Bangalore}{Lead Engineer, AVI Solutions Team}{2014-2016}
		\vspace{1mm}
		\begin{itemize}
				% Project-1 
				\resitem[]{\faCircleO \hspace{1mm}Project: Histogram of Oriented Gradients for Pedestrian Detection}
				\begin{itemize}
						\resitem{Development: Languages \& Tools used - \CC, OpenCV, SVMLight}
				\end{itemize}
				\begin{itemize}
						\resitem{Aim of this project is to:}
						\begin{itemize}
								\resitem{Demonstrate my understanding in Support Vector Machines by applying to a computer vision problem}
								%\resitem{Dataset: INRIA dataset (Window Size - 160x90)}
								%\resitem{HOG features are computed using OpenCV function}
								%\resitem{SVMLight is used to train the feature extracted}
								%\resitem{Detection is done using OpenCV function}
								%\resitem{False positives are used to train the model again}
								%\resitem{Precision/recall on test set: 97\%/86\%}
						\end{itemize}
				\end{itemize}

				\vspace{1mm}
				% Project-2 
				\resitem[]{\faCircleO \hspace{1mm}Project: Combining Sketch and Tone for Pencil Drawing Production. 
					\faWindows \hspace{1mm} \href{https://github.com/nrupatunga/Pencil-Sketch/releases/tag/v1.0-beta}{\emph{Software}}
					\faGithub \hspace{1mm} \href{https://github.com/nrupatunga/Color-Pencil-Sketch}{\emph{Code}}}
				\begin{itemize}
						\resitem{Development: Languages \& Tools used - \CC, OpenCV, QT}
						\begin{itemize}
								\resitem{A system to produce pencil drawings from natural images.}
								\resitem{This system mimicks human style of pencil drawing}
								\resitem{Designed a GUI using QT}
						\end{itemize}
				\end{itemize}
				\vspace{1mm}
				% Project-3
				\resitem[]{\faCircleO \hspace{1mm}Project: Auto Image Enhancement (Galaxy S6 onwards)}
				\begin{itemize}
						\resitem{Design and development: Languages used - C, Matlab}
						\begin{itemize}
								\resitem{Algorithm for detection of low-light/backlight images}
								\resitem{Algorithm for detection of poorly lit faces in an image}
								\resitem{Colorfulness measurement in natural images}
						\end{itemize}
						\resitem{Complete architecture design of Auto Image Enhancement Engine }
						\resitem{Complete JNI framework design \& development for communicating between application and engine}
				\end{itemize}
				\vspace{1mm}
				% Project-4
				\resitem[]{\faCircleO \hspace{1mm}Project: Touch Focus (Galaxy S5 onwards)}
				\begin{itemize}
						\resitem{Complete JNI framework design \& development for communicating between application and engine}
				\end{itemize}
		\end{itemize}
		\vspace{1mm}
	\item
		\ressubheading{Samsung R\&D India}{Bangalore}{Senior Software Engineer, Multimedia Solutions Team}{2012-2014}
		\begin{itemize}
				\resitem[]{\faCircleO \hspace{1mm}Project: Photo Editor, Best Photo.}
				\begin{itemize}
						\resitem{Design and development: Red eye correction algorithm. GUI developed using Matlab GUIDE for quick demo }
						\resitem{Design and development: Measurement of blur in an image. Algorithms implemented from two IEEE papers. 
							Languages used:\CC}
						\resitem{Implementation of bilinear resizer module for less memory architecture - Insert emoticon effect in Photo Editor. 
							Languages used: C}
						\resitem{Optimization of Photo Editor effects using POSIX threads}
				\end{itemize}
		\end{itemize}

\end{itemize}

\resheading{Pet Projects}
\begin{enumerate}
	\item Implementation of Canny Edge Detector. Languages \& Tools used - \CC, OpenCV.
		\faGithub \hspace{1mm}\href{https://github.com/nrupatunga/Canny-Edge-Detector}{\emph{Code}}
	\item Implementation of Bilateral filter. Languages \& Tools used - \CC, OpenCV.
		\faGithub \hspace{1mm}\href{https://github.com/nrupatunga/Bilateral-Filter}{\emph{Code}}
	\item QT based GUI Application for experimenting Sobel \& Canny Edge Detectors. Languages \& Tools used - \CC, OpenCV, QT. 
		\faWindows \hspace{1mm}\href{https://github.com/nrupatunga/Computer-Vision-Tool}{\emph{Software}}
\end{enumerate}

\resheading{Relevant Coursework - IISc, Bangalore}
\begin{description}
	\item[Signal Processing Courses:]
		Digital Image Processing, DSP System Design, Biomedical Signal Processing, Speech Information Processing
	\item[Mathematical Courses:]
		Linear Algebra, Probability \& Random Process, Detection \& Estimation Theory, Mathematics for Electrical Engineers
\end{description}

\resheading{Skills}
\begin{description}
	\item[Languages:]
		C/\CC, MATLAB, Python, QT, Android JNI
	\item[Tools:]
		Microsoft Visual Studio, Eclipse, Android NDK, Vim
	\item[Miscellaneous:]
		Excellent troubleshooting and debugging skills
\end{description}

\end{document}
