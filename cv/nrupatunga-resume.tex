% resume.tex
%
% (c) 2002 Matthew Boedicker <mboedick@mboedick.org> (original author) http://mboedick.org
% (c) 2003-2007 David J. Grant <davidgrant-at-gmail.com> http://www.davidgrant.ca
% (c) 2007-2014 Todd C. Miller <Todd.Miller@courtesan.com> http://www.courtesan.com/todd
% (c) 2016 Nrupatunga <https://nrupatunga.github.io/>
%
% This work is licensed under the Creative Commons Attribution-ShareAlike 3.0 Unported License. To view a copy of this license, visit http://creativecommons.org/licenses/by-sa/3.0/ or send a letter to Creative Commons, 171 Second Street, Suite 300, San Francisco, California, 94105, USA.

\documentclass[letterpaper,11pt]{article}

%-----------------------------------------------------------
\usepackage[empty]{fullpage}
\usepackage{color}
\usepackage{xcolor}
\usepackage[colorlinks = true,
            linkcolor = blue,
            urlcolor  = blue,
            citecolor = blue,
            anchorcolor = blue]{hyperref}
\usepackage{hyperref}
\usepackage{xspace}
\usepackage{verbatim}
\usepackage{marvosym}
\usepackage{fontspec}
\usepackage{fontawesome}
\definecolor{mygrey}{gray}{0.80}
\raggedbottom
\raggedright
\setlength{\tabcolsep}{0in}

% Adjust margins to 0.5in on all sides
\addtolength{\oddsidemargin}{-0.5in}
\addtolength{\evensidemargin}{-0.5in}
\addtolength{\textwidth}{1.0in}
\addtolength{\topmargin}{-0.5in}
\addtolength{\textheight}{1.0in}

%-----------------------------------------------------------
%Custom commands
\def\CC{{C\nolinebreak[4]\hspace{-.05em}\raisebox{.4ex}{\tiny\bf ++}}}
%\newcommand{\CC}{C\nolinebreak\hspace{-.05em}\raisebox{.4ex}{\tiny\bf +}\nolinebreak\hspace{-.10em}\raisebox{.4ex}{\tiny\bf +}}
\newcommand*{\Cpp}{C\ensuremath{++}\xspace}
\newcommand*{\phoneB}{{\fontspec{symbola.ttf}\symbol{"260E}}}
\newcommand{\resitem}[1]{\item #1 \vspace{-2pt}}
\newcommand{\resheading}[1]{{\large \colorbox{mygrey}{\begin{minipage}{\textwidth}{\textbf{#1 \vphantom{p\^{E}}}}\end{minipage}}}}
\newcommand{\ressubheading}[4]{
	\begin{tabular*}{7.0in}{l@{\extracolsep{\fill}}r}
		\textbf{#1} & #2 \\
		\textit{#3} & \textit{#4} \\
	\end{tabular*}\vspace{-6pt}}
%-----------------------------------------------------------

\begin{document}

\begin{tabular*}{7.5in}{l@{\extracolsep{\fill}}r}
	\textbf{\Large Nrupatunga}  & \faMobile \hspace{1mm}\emph{+91-9972716433}\\
	&\faGlobe \hspace{1mm} \href{https://nrupatunga.github.io/}{\emph{https://nrupatunga.github.io/}}\\
	&\faEnvelope \hspace{1mm}\emph{nrupatunga.tunga@gmail.com} \\
	%&  \faLinkedin \hspace{1mm}\emph{https://in.linkedin.com/in/nrupatunga} \\
	%&\faGithub \hspace{1mm} \href{https://github.com/nrupatunga}{\emph{https://github.com/nrupatunga}} \\
\end{tabular*}
\\

\vspace{0.1in}
\resheading{Research Interest}
\begin{itemize}
		\resitem{My current interest and focus is on applying deep learning for scene understanding which includes scene classification, object detection, recognition and segmentation.}
		%\resitem{I have strong interest towards deep network compression and reverse engineering these networks to understand what each layers in the network have learnt from the scene for the required task.}
\end{itemize}

\resheading{Education}
\begin{itemize}
	\item
		\ressubheading{Indian Institute of Science}{Bangalore, Karnataka}{Master of Engineering in Signal Processing}{2010 - 2012}
		\begin{itemize}
				\resitem{Master Thesis: Complex Network Approach for Analysis of Biomedical signals}
				\resitem{CGPA: 5.8/8.0}
		\end{itemize}
	\item
		\ressubheading{Sri Jayachamarajendra College of Engineering}{Mysore, Karnataka}{Bachelor of Engineering in Electronics and Communication }{2005 - 2009}
\end{itemize}

\resheading{Work Experience \& Projects}
\begin{itemize}
	\item
		\ressubheading{Samsung R\&D India}{Bangalore}{Technical Lead, Media Analytics and Recognition Team}{2012-Present}
		\vspace{4mm}
		\begin{itemize}
				% Project-1 
				\resitem[]{\faCircleO
					\hspace{1mm}\textbf{\emph{Deep Convolutional Network for Food Recognition}}\hspace{1mm}}
				\begin{itemize}
						\resitem{Trained Squeezenet model for real time inference on Android devices}
						\resitem{Trained Resnet-50, Resnet-101, Resnet-152 models with data augmentation and tweak to the model architecture to improve the recognition accuracy}
						\resitem{Languages \& Tools used - Python, Caffe}
				\end{itemize}
				\vspace{2.5mm}
				\resitem[]{\faCircleO
					\hspace{1mm}\textbf{\emph{Deep Convolutional Network for Image Aesthetics}}\hspace{1mm}}
				\begin{itemize}
						\resitem{Trained 2-column VGG-16 model and GoogleNet model with data augmentation including data oversampling and multiple input crops}
						\resitem{Application developed to classify a given image into high and low quality}
						\resitem{Languages \& Tools used - Python, PyQt, Caffe}
				\end{itemize}
				\vspace{2.5mm}
				% Project-2
				\resitem[]{\faCircleO
					\hspace{1mm}\textbf{\emph{Fully Convolutional Network for Segmentation of Sky and Non-sky regions}}\hspace{1mm}
					\faGlobe \hspace{1mm} \href{https://nrupatunga.github.io/fcn-segmentation/}{\emph{blog}}}
				\begin{itemize}
						\resitem{Trained fully convolutional VGG-16 model. Sky segmentation map used as prior for horizon detection in an image}
						\resitem{Languages \& Tools used - Python, Caffe}
				\end{itemize}
				\vspace{2.5mm}
				% Project-2
				% Project-3
				\resitem[]{\faCircleO \hspace{1mm}\textbf{\emph{Nearest Neighbor Image Retrieval using GIST}}\hspace{1mm}
					\faGithub \hspace{1mm} \href{https://github.com/nrupatunga/GIST-global-Image-Descripor}{\emph{code}}}
				\begin{itemize}
						\resitem{Developed code for extracting GIST descriptor for images}
						\resitem{Demonstrated how GIST descriptor can be used for detection of duplicate images in Gallery}
						\resitem{Languages \& Tools used - \CC, OpenCV, MATLAB}
				\end{itemize}
				% Project-1 
				\vspace{2.5mm}
				% Project-2 
				\resitem[]{\faCircleO \hspace{1mm}\textbf{\emph{Combining Sketch and Tone for Pencil Drawing Production}}\hspace{1mm}
					\faGithub \hspace{1mm} \href{https://github.com/nrupatunga/Color-Pencil-Sketch}{\emph{code}}}
				\begin{itemize}
						\resitem{Code developed for color pencil sketch effect for images which mimicks human style of pencil drawing}
						\resitem{Application deveopment for Color Pencil Sketch}
						\resitem{Languages \& Tools used - \CC, OpenCV, QT}
				\end{itemize}
				\vspace{15mm}
				% Project-3
				\resitem[]{\faCircleO \hspace{1mm}\textbf{\emph{One Touch Auto Image Enhancement (Galaxy S6 onwards)}}}
				\begin{itemize}
						\resitem{Developed algorithm for detection of low-light/backlight images}
						\resitem{Developed algorithm for detection of poorly lit faces in an image}
						\resitem{Complete architecture design of auto image enhancement engine }
						\resitem{Complete JNI framework design \& development for communicating between application and engine}
						\resitem{Languages \& Tools used - C, Matlab}
				\end{itemize}
				\vspace{2.5mm}
				\resitem[]{\faCircleO \hspace{1mm}\textbf{\emph{Photo Editor, Best Photo}}}
				\begin{itemize}
						\resitem{Developed red eye correction algorithm. GUI developed using Matlab GUIDE for quick demo }
						\resitem{Implemented image blur detection and ranking algorithms}
						\resitem{Implemented bilinear resizer module for less memory architecture in Photo Editor}
						\resitem{Optimization of Photo Editor effects using POSIX threads}
						\resitem{Languages \& Tools used - C, \CC, Matlab}
				\end{itemize}
				\vspace{2.5mm}
				\resitem[]{\faCircleO \hspace{1mm}\textbf{\emph{Touch Focus (Galaxy S5 onwards)}}}
				\begin{itemize}
						\resitem{Complete JNI framework design \& development for communicating between application and engine}
				\end{itemize}
		\end{itemize}

\end{itemize}

\resheading{Hobby Projects}
\begin{enumerate}
	\item Implementation of RNN and LSTM from scratch for character prediction. Languages \& Tools used - Python, Numpy
		\faGithub \hspace{1mm}\href{https://github.com/nrupatunga/char-rnn}{\emph{code-rnn}}
		\faGithub \hspace{1mm}\href{https://github.com/nrupatunga/multi-layer-lstm}{\emph{code-lstm}}
	\item Trained a SVM model for Pedestrain detection using Histogram of Oriented Gradients feature. Languages \& Tools used - \CC, OpenCV, Python
		\faGithub \hspace{1mm}\href{https://github.com/nrupatunga/Pedestrain-Detection-using-Histogram-of-Oriented-Gradients}{\emph{code}}
	\item QT based GUI Application for experimenting edge detectors such as Sobel \& Canny, blurring filters such as homogeneous, median, Gaussian \& bilateral.
		Languages \& Tools used - \CC, OpenCV, QT
		\faWindows \hspace{1mm}\href{https://github.com/nrupatunga/Computer-Vision-Tool}{\emph{software}}
\end{enumerate}

\resheading{Recognition}
\begin{enumerate}
	\item Awarded Employee of the month - Jan 2016
	\item Awarded Galaxy S5 for the effort in success of Touch Focus USP
\end{enumerate}

\resheading{Relevant Coursework}
\begin{description}
	\item[Deep Learning:]
		Learning from Data (Caltech), Machine Learning (Stanford), UFLDL (Stanford), Stanford CS231 course
	\item[Signal Processing Courses:]
		Digital Image Processing, DSP System Design, Biomedical Signal Processing, Speech Information Processing
	\item[Mathematical Courses:]
		Linear Algebra, Probability \& Random Process, Detection \& Estimation Theory, Mathematics for Electrical Engineers
\end{description}

\resheading{Skills}
\begin{description}
	\item[Programming Languages:]
		C, \CC, MATLAB, Python
	\item[Tools \& Framework:]
		Caffe Deep Learning Framework, Microsoft Visual Studio, QT, Eclipse, Android JNI/NDK
	\item[Work Productivity Tools:] Vim, tmux
\end{description}

\end{document}
